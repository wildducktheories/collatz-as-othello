\documentclass[12pt]{article}
\usepackage{amsmath, amssymb, amsfonts, amsthm}
\usepackage{hyperref}
\usepackage{geometry}
\geometry{margin=1in}

\newtheorem{theorem}{Theorem}[section]
\newtheorem{lemma}[theorem]{Lemma}
\newtheorem{proposition}[theorem]{Proposition}
\newtheorem{corollary}[theorem]{Corollary}
\theoremstyle{definition}
\newtheorem{definition}[theorem]{Definition}
\newtheorem{example}[theorem]{Example}
\theoremstyle{remark}
\newtheorem{remark}[theorem]{Remark}

\title{Visualizing Collatz Cycles on the $x$ vs.\ $\Delta k$ Plot}
\author{Jon Seymour}
\date{\today}

\begin{document}

\maketitle

\begin{abstract}
Every Collatz-like cycle satisfies the identity $x \cdot d = q \cdot k$, where
$d = h^e - g^o$ is determined solely by the cycle's step counts and base parameters,
$k$ is a monomial sum encoding the cycle's internal structure, and $q$ is a reduced
cofactor.  We introduce a reference value $\hat{k} = (g^o - h^o)/(g-h)$ representing the
minimum achievable $k$ for a cycle with $o$ odd steps, and write $k = \hat{k} + \Delta k$
with $\Delta k \geq 0$.  Substituting into the cycle identity yields an affine equation
expressing $x$ as a linear function of $\Delta k$.  We derive tight bounds
$0 \leq \Delta k \leq \Delta k_{\max}$, where
$\Delta k_{\max} = (h^{e-o} - 1)(\hat{k} - g^{o-1})$, and observe that every odd-step
cycle element must correspond to an integer lattice point on this line within those bounds.  The
resulting $x$ vs.\ $\Delta k$ plot provides a compact visualization of the solution space
for cycles of a given type $(g, h, o, e)$.
\end{abstract}

\section{Introduction}

A \emph{Collatz-like} dynamical system is defined by two positive integers $g$ (odd) and
$h \geq 2$ with $\gcd(g, h) = 1$.  The map $x \mapsto (gx + 1)/h^{v_h(gx+1)}$ is applied
to odd positive integers, where $v_h(n)$ denotes the largest power of $h$ dividing $n$.
The classical $3x+1$ system corresponds to $g = 3$, $h = 2$.  A \emph{cycle} is a finite
set of odd integers $\{x_1, x_2, \ldots, x_o\}$ closed under this map, with each $x_j$
contributing one odd step and $e_j \geq 1$ even steps before reaching $x_{j+1}$.

It is well known that every such cycle satisfies the polynomial identity
\begin{equation}\label{eq:cycle-identity}
  x \cdot d(g, h) = q \cdot k(g, h),
\end{equation}
where $x$ is any element of the cycle, and $d$, $k$, and $q$ are derived quantities
defined in Section~\ref{sec:background}.

The central observation of this paper is that, for fixed $(g, h, o, e)$, the identity
\eqref{eq:cycle-identity} defines an \emph{affine function} of a deviation parameter
$\Delta k = k - \hat{k}$, where $\hat{k}$ is the minimum possible value of $k$ for a cycle
with $o$ odd steps.  All valid cycles must therefore appear as integer lattice points on a
single line in the $(\Delta k, x)$ plane, within a computable interval of $\Delta k$ values.
This gives rise to the \emph{$x$ vs.\ $\Delta k$ plot}: a visualization that collapses the
$o$-dimensional family of cycle configurations onto a one-dimensional affine constraint.

\section{Background}\label{sec:background}

Fix $g$ (odd, $g \geq 3$) and $h \geq 2$ with $\gcd(g, h) = 1$.  Let $o \geq 1$ denote
the number of odd steps in a cycle and $e \geq 1$ the total number of even steps, with the
even steps distributed as $e_1, e_2, \ldots, e_o \geq 1$ (so $e = \sum_{j=1}^o e_j$).

\subsection{The Polynomial Representation}

Following \cite{othello-board}, a cycle is encoded by a positive integer $p$ whose binary
structure records the sequence of odd and even steps.  Parsing the binary expansion of $p$
from the least-significant bit yields $o$ monomials.  The $j$-th monomial (for
$j = 0, 1, \ldots, o-1$) has $g$-exponent $o - 1 - j$ and $h$-exponent
\begin{equation}
  i_j = \sum_{l=1}^{j} e_l,
\end{equation}
the cumulative even-step count before the $(j+1)$-th odd step.  In particular,
\begin{equation}\label{eq:isequence}
  0 = i_0 < i_1 < \cdots < i_{o-1} < e,
\end{equation}
where $i_0 = 0$ because every cycle element is odd (no even steps precede the first odd
step in any traversal), and $i_{o-1} = e - e_o < e$ because at least one even step $e_o
\geq 1$ follows the last odd step.

\begin{definition}
  The \emph{monomial sum} of a cycle is
  \begin{equation}
    k = k(g, h) = \sum_{j=0}^{o-1} g^{o-1-j}\, h^{i_j}.
  \end{equation}
  The \emph{difference polynomial} is
  \begin{equation}
    d = d(g, h) = h^e - g^o.
  \end{equation}
\end{definition}

\begin{remark}
  For a cycle to have positive elements we need $d > 0$, i.e.\ $h^e > g^o$, equivalently
  $e > o \log_h g$.  This requires $e > o$ when $g > h$.
\end{remark}

\begin{definition}
  Given $k$ and $d$ as above, set $\gamma = \gcd(k, d)$ and define
  \begin{equation}
    x_0 = \frac{k}{\gamma}, \qquad q = \frac{d}{\gamma}.
  \end{equation}
\end{definition}

The cycle identity \eqref{eq:cycle-identity} then holds with $x = x_0$:
\begin{equation*}
  x_0 \cdot d = \frac{k}{\gamma} \cdot d = k \cdot \frac{d}{\gamma} = q \cdot k. \qquad\checkmark
\end{equation*}

\section{The Reference Value $\hat{k}$ and the Deviation $\Delta k$}

\begin{definition}
  The \emph{reference monomial sum} for a cycle with $o$ odd steps is
  \begin{equation}
    \hat{k} = \hat{k}(g, h, o) = \frac{g^o - h^o}{g - h} = \sum_{j=0}^{o-1} g^{o-1-j}\, h^j.
  \end{equation}
  For $g - h = 1$ (in particular for $g = 3$, $h = 2$) this simplifies to
  $\hat{k} = g^o - h^o$.
\end{definition}

The value $\hat{k}$ is realised by the configuration $i_j = j$ for all $j$, i.e.\ the
monomials occupy consecutive rows $0, 1, \ldots, o-1$.  This configuration corresponds to
cycles in which every odd step except the last is followed by exactly one even step.  We
now show this is the minimum possible $k$.

\begin{proposition}\label{prop:khat-minimum}
  For any valid cycle with $o$ odd steps, $k \geq \hat{k}$, with equality if and only if
  $i_j = j$ for all $j = 0, 1, \ldots, o-1$.
\end{proposition}

\begin{proof}
  Since $i_0 = 0$ and $i_0 < i_1 < \cdots < i_{o-1}$, we have $i_j \geq j$ for all $j$.
  As $h \geq 2$, each term satisfies $g^{o-1-j}\,h^{i_j} \geq g^{o-1-j}\,h^j$, with
  equality iff $i_j = j$.  Summing over $j$ gives $k \geq \hat{k}$.
\end{proof}

\begin{definition}
  The \emph{deviation} is $\Delta k = k - \hat{k} \geq 0$.
\end{definition}

\section{The Affine Equation and Its Bounds}

\subsection{The Affine Equation}

Substituting $k = \hat{k} + \Delta k$ into the cycle identity $x \cdot d = q \cdot k$
yields
\begin{equation}\label{eq:affine}
  x = \underbrace{\frac{q\,\hat{k}}{d}}_{\text{intercept}} +
      \underbrace{\frac{q}{d}}_{\text{slope}}\,\Delta k.
\end{equation}
Thus, for fixed $(g, h, o, e)$ — and hence fixed $d$, $\hat{k}$, and $q$ — the cycle
element $x$ is an affine function of $\Delta k$ with slope $q/d$ and intercept
$q\hat{k}/d$.

\subsection{Upper Bound on $\Delta k$}

\begin{proposition}\label{prop:deltamax}
  For any valid cycle with $o$ odd steps and $e$ even steps,
  \begin{equation}\label{eq:deltamax}
    \Delta k \leq \Delta k_{\max} = \bigl(h^{e-o} - 1\bigr)\bigl(\hat{k} - g^{o-1}\bigr).
  \end{equation}
\end{proposition}

\begin{proof}
  We maximise $k$ subject to $i_0 = 0$ and $0 = i_0 < i_1 < \cdots < i_{o-1} < e$.
  The maximum is attained by setting $i_0 = 0$ and packing the remaining monomials
  ($j = 1, \ldots, o-1$) into the highest possible consecutive rows ending at $e - 1$:
  \begin{equation*}
    i_j = e - o + j, \quad j = 1, \ldots, o-1.
  \end{equation*}
  (Note $i_1 = e - o + 1 > 0$ since $e > o$, and $i_{o-1} = e - 1 < e$.)  This gives
  \begin{align*}
    k_{\max} &= g^{o-1} \cdot h^0 + \sum_{j=1}^{o-1} g^{o-1-j}\, h^{e-o+j} \\
             &= g^{o-1} + h^{e-o+1} \sum_{j=1}^{o-1} g^{o-1-j}\, h^{j-1} \\
             &= g^{o-1} + h^{e-o+1} \cdot \frac{\hat{k} - g^{o-1}}{h},
  \end{align*}
  where the last step uses the identity $\hat{k} = g^{o-1} + h \sum_{j=1}^{o-1}
  g^{o-1-j} h^{j-1}$.  Therefore
  \begin{equation*}
    \Delta k_{\max} = k_{\max} - \hat{k}
    = g^{o-1} + h^{e-o}\,(\hat{k} - g^{o-1}) - \hat{k}
    = (h^{e-o} - 1)(\hat{k} - g^{o-1}). \qedhere
  \end{equation*}
\end{proof}

\begin{remark}
  For $o = 1$ we have $\hat{k} = 1 = g^{o-1}$, so $\Delta k_{\max} = 0$: there is only
  one monomial, pinned at $i_0 = 0$, and $k = 1$ always.
\end{remark}

\begin{remark}
  The bound $\Delta k \leq \Delta k_{\max}$ applies specifically to \emph{odd} cycle
  elements, i.e.\ those whose $p$-value is odd (equivalently, elements $x$ to which the
  operation $x \mapsto gx + q$ is applied).  Even $p$-value elements (halving steps) may
  possess $\Delta k$ values that fall outside the interval $[0, \Delta k_{\max}]$; the
  bound is not intended to constrain them.  In particular, \emph{forced} cycle elements
  --- those for which $p \bmod 2 \neq x \bmod 2$ --- are even $p$-value elements whose
  $x$-value is odd, and their $\Delta k$ values need not respect this bound.
\end{remark}

\subsection{Summary of Bounds}

Combining Propositions \ref{prop:khat-minimum} and \ref{prop:deltamax}:
\begin{equation}\label{eq:bounds}
  0 \leq \Delta k \leq (h^{e-o} - 1)\,(\hat{k} - g^{o-1}).
\end{equation}
Substituting into \eqref{eq:affine}, the cycle element $x$ is constrained to the interval
\begin{equation}
  \frac{q\,\hat{k}}{d} \;\leq\; x \;\leq\;
  \frac{q\,\hat{k}}{d} + \frac{q}{d}\,(h^{e-o}-1)(\hat{k}-g^{o-1}).
\end{equation}

\section{The $x$ vs.\ $\Delta k$ Plot}

The affine constraint \eqref{eq:affine} and the bounds \eqref{eq:bounds} together define a
line segment in the $(\Delta k, x)$ plane.  The \emph{$x$ vs.\ $\Delta k$ plot} for a
given $(g, h, o, e)$ displays this line segment together with the integer lattice points
it contains, each of which is a candidate cycle element.

\begin{definition}
  A \emph{lattice point} on the $x$ vs.\ $\Delta k$ line is a pair
  $(\Delta k, x) \in \mathbb{Z}_{\geq 0} \times \mathbb{Z}_{>0}$ satisfying
  \eqref{eq:affine} and \eqref{eq:bounds}.
\end{definition}

\begin{remark}
  Not every lattice point corresponds to an actual cycle: the value $x = k/\gamma$ must be
  a positive odd integer, and the corresponding $p$ must encode a valid binary cycle
  structure.  The plot therefore provides necessary but not sufficient conditions for a
  cycle.
\end{remark}

The significance of the plot is as follows.  For a fixed $(g, h, o, e)$, the slope $q/d$
and intercept $q\hat{k}/d$ are determined.  Any cycle of that type must lie on the line.
The finite interval $[0, \Delta k_{\max}]$ bounds how far an odd-step cycle element can
deviate from the minimum-$k$ configuration.  Cycles with small $\Delta k$ are ``close to natural'' in the
sense that their monomials are spread uniformly across low rows $0, 1, \ldots, o-1$;
cycles with $\Delta k$ near $\Delta k_{\max}$ have their free monomials packed near row
$e - 1$.

\begin{example}[The $5x+1$ cycle with $x_0 = 17$]
  The cycle $17 \to 43 \to 27 \to 17$ under the $5x+1$ map ($g = 5$, $h = 2$) has
  $o = 3$ odd steps and even-step counts $e_1 = 1$, $e_2 = 3$, $e_3 = 3$, giving
  $e = 7$.  The monomial $h$-exponents are $i_0 = 0$, $i_1 = 1$, $i_2 = 4$, so
  \begin{align*}
    k &= 5^2 \cdot 2^0 + 5^1 \cdot 2^1 + 5^0 \cdot 2^4 = 25 + 10 + 16 = 51, \\
    d &= 2^7 - 5^3 = 128 - 125 = 3, \\
    \gamma &= \gcd(51, 3) = 3, \quad x_0 = 51/3 = 17, \quad q = 3/3 = 1.
  \end{align*}
  The reference value and deviation are
  \begin{align*}
    \hat{k} &= (5^3 - 2^3)/(5-2) = 117/3 = 39, \\
    \Delta k &= 51 - 39 = 12.
  \end{align*}
  The upper bound is
  \begin{equation*}
    \Delta k_{\max} = (2^{7-3} - 1)(39 - 25) = 15 \times 14 = 210.
  \end{equation*}
  The affine line has slope $q/d = 1/3$ and intercept $q\hat{k}/d = 39/3 = 13$, giving
  \begin{equation*}
    x = 13 + \tfrac{1}{3}\,\Delta k.
  \end{equation*}
  At $\Delta k = 12$: $x = 13 + 4 = 17$. \checkmark\quad The cycle lies well within
  $[0, 210]$.  Integer lattice points on this line require $\Delta k \equiv 0 \pmod{3}$.
\end{example}

\section{Conclusion}

We have shown that the cycle identity $x \cdot d = q \cdot k$ can be rewritten as the
affine equation $x = q\hat{k}/d + (q/d)\,\Delta k$, where $\Delta k = k - \hat{k}$
measures the deviation of the monomial sum from its minimum value
$\hat{k} = (g^o - h^o)/(g-h)$.  The key structural constraint is that, since all cycle
elements are odd, the first monomial $h$-exponent is always $i_0 = 0$, while the last is
at most $i_{o-1} = e - 1$.  This pins the minimum at $\hat{k}$ and yields the tight upper
bound $\Delta k_{\max} = (h^{e-o} - 1)(\hat{k} - g^{o-1})$.

The $x$ vs.\ $\Delta k$ plot visualises this constraint: all cycles of a given type
$(g, h, o, e)$ collapse onto a single line, and valid cycle elements must be integer
lattice points within the computable bounds.  This provides an efficient way to enumerate
candidate cycles and to understand the solution space of the Collatz-like cycle equation.

\begin{thebibliography}{9}

\bibitem{othello-board}
  J.~Seymour,
  \emph{Othello Board Analogy for Collatz-Type Cycles},
  unpublished manuscript, 2025.

\bibitem{affine-blocks}
  J.~Seymour,
  \emph{Affine Block Structures in Collatz Sequences},
  unpublished manuscript, 2025.

\end{thebibliography}

\end{document}
